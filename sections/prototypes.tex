\documentclass[../design-document.tex]{subfiles}
\begin{document}

\subsection{ProtoWADT v0}
The first prototype was created to explore controlling Docker containers programmatically and gain familiarity with Django. This prototype was timeboxed to a 1-day sprint to prohibit overanalysis and decision paralysis.

The scope was intentionally basic and focused on only programmatically controlling one container, instead of the usual 7-10. It ran local-only,
depended on a specific vulnerable app (DVWA), and was intended to be thrown away.

ProtoWADT v0 was also a vital exploration of some of the invalid assumptions the team had made previously about our tech stack. Different libraries were depended on than originally predicted as a result of this exploration. This prototype also began the thinking which led to further design exploration and investigation of hidden complexity in the next prototype.

The prototype's definition of done required that at least the core states of Docker containers could be manipulated through API endpoints. Start, Restart, Logs, Stop, and Down were initially required.

A crucial oversight of this prototype was the lack of a database, which hid further problems with state and storage that were not discovered until the second prototype. The decision to store important data in memory for a local-only demo sidestepped issues of data corruption, handling asynchronous connections, and scaling that are discussed in more detail in \ref{api-considerations} and \ref{container-identity-management}.

\subsubsection{Programmatic Control}
The first goal of v0 was to programmatically control Docker containers with a suitable programming language. For Python, two main libraries exist that fit this purpose: python-on-whales and docker-py. Broadly, docker-py is intended for applications that run as a single container, and python-on-whales is intended for applications that run as a group of containers with Docker Compose as well as single-container applications. Since DVWA specifically uses Docker Compose to run multiple containers, python-on-whales was selected for the prototype.

First, a short demo was created that essentially listed and used the main parts of the python-on-whales API that the project would depend on. python-on-whales has essentially a one-to-one correspondence beteween its API and the Docker CLI command structure, meaning that finding these endpoints was often as easy as typing the command with Python syntax.

\begin{figure}[H]
	\centering
	\begin{minted}{console}
	docker compose up --detach
	\end{minted}
	\caption{\texttt{docker} command syntax}
	\label{fig:docker-compose-up-cli}
\end{figure}

\begin{figure}[H]
	\centering
	\begin{minted}{python}
	docker.compose.up(detach=True)
	\end{minted}
	\caption{python-on-whales API}
	\label{fig:docker-compose-up-python}
\end{figure}


\subsubsection{Django}

\subsection{ProtoWADT v1}


\end{document}
